\section{Auswertung}

\subsection{Strahlungsbelastung}

Da für das Experiment radioaktive Präparate benutzt werden, scheint eine
Abschätzung der maximalen, persönlichen Belastung sinnvoll.

Folgende Annahmen wurden getroffen:
\begin{itemize}
  \item Die betreffende Person wiegt $m = \SI{80}{\kilo\gram}$.
  \item Die Einwirkdauer beträgt jeweils $s = \SI{12}{\hour}$.
  \item Die Aktivität $A$ aller 3 Präparate übersteigt jeweils nicht
        \SI{4e5}{\becquerel}.
  \item Die Halbzeit $t$ aller 3 Präparate wird mit dem maximalen Wert, 30 Jahren,
        abgeschätzt.
  \item Seit der Vermessung der Präparate sind aufgerundet $u = $ 3 Jahre vergangen.
  \item \SI{100}{\percent} der Strahlung wird aufgenommen.
  \item Der Strahlungsgewichtungsfaktor für γ-Strahlung beträgt 1.
  \item Die γ-Quanten haben eine maximale Energie von
        $E = \SI{0.7}{\mega\eV} = \SI{1.12e-13}{\joule}$
  \item Es gibt $n = 3$ Proben.
\end{itemize}
Damit berechnet sich die radioaktive Belastung zu
\begin{equation}
  s \left(\frac{1}{2}\right)^{u/t} n E A / m = \SI{0.07}{\milli\sievert}
\end{equation}
und macht somit bei einer durchschnittlichen Strahlungsbelastung in Deutschland von
\SI{4.3}{\milli\sievert} pro Jahr rund \SI{2}{\percent} aus.

\subsection{Signalbetrachtung}
In unserem Experiment wurde ein Szintillationsdetektor, an den ein
Photomultiplier angeschlossen war, verwendet, um ankommende Photonen in ein
messbares elektrisches Signal umzuwandeln. Dieses Signal wurde anschließend
durch einen Vorverstärker und zusätzlich noch durch einen Hauptverstärker
geleitet. Auch wenn sich in der Nähe des Detektors keine radioaktive Quelle
befand wurden trotzdem Signale gemessen, was hauptsächlich durch das thermisch
bedingtes Rauschen der beteiligten Komponenten verursacht wurde. Diese
Rausch-Signale konnten beobachtet und auf ihre Form hin untersucht werden.

Sowohl das Signal des Vorverstärkers als auch des Hauptverstärkers kann in
\fref{signalform} betrachtet werden.
\begin{figure}[htb]
      \centering
      \includegraphics[width=1\columnwidth,keepaspectratio]{signalform}
      \caption{Standbild des Digital-Oszilloskops zur Signalform-Betrachtung}
      \label{fig:signalform}
\end{figure}

Man kann erkennen, dass die schwarze Linie dem Vorverstärkersignal entspricht,
was dadurch deutlich wird, dass die Achseneinteilung (an Channel 1) bei gerade
einmal \si{5}{\milli\volt} pro Kästchen lag, während das noch einmal verstärkte
Hauptverstärkersignal (grüne Linie) bei einer Achseneinteilung von
\si{2}[{\volt} aufgetragen wurde. Offenbar wird das Signal im Hauptverstärker
noch einmal geglättet, denn dessen Signal entspricht einer Gauß-Kurve, wobei im
Gegensatz dazu das Vorverstärkersignal sprungartig ansteigt, dann aber viel
Zeit braucht, um wieder auf das Niveau vor der Messung abzufallen (man kann
grob abschätzen, dass das Vorverstärkersignal ungefähr
$6\cdot20\si{\micro\second} = 120\si{\micro\second}$ für den Abfall benötigt,
wohingegen das Hauptverstärkersignal nur etwa 10\si{\micro\second} braucht.)

\subsection{Latenzzeit der Messtechnik}

\begin{figure}[htb]
      \centering
      \includegraphics[width=1\columnwidth,keepaspectratio]{messverzoegerung}
      \caption{Standbild des Digital-Oszilloskops zur Messung der Latenzzeit}
      \label{fig:latenz}
\end{figure}

Durch die Verwendung des Hauptverstärkers kann das Signal mit höherer Präzision
gemessen werden, jedoch muss man dafür eine zeitliche Verzögerung $Δt$ in Kauf
nehmen. Um jene zu bestimmen, wird der Trigger des Oszilloskops auf den
Vorverstärker gestellt, die Schwellspannung für das Triggern auf einen Wert
hochgeregelt, der nur bei Eintreffen eines Signals erreicht wird und nicht
durch die permanente Fluktuation, und dann wurde das Bild manuell angehalten, so
dass im Ergebnis ein Standbild wie in \fref{latenz} exportiert werden kann.
Hierfür haben wir die Zeitskala kleiner eingestellt, um bessere Messungen
vornehmen zu können. Das Rädchen, mit dem die Ablesehilfe verschoben werden
konnte, ließ sich in der gewählten Einstellung in
0,1\si{\micro\second}-Schritten verstellen. Damit ist jeder einzelne Messwert
mit diesem zufälligen Fehler behaftet. Da in diesen Schritten eindeutig
erkennbar war, bei welcher Schieber-Position das Maximum der
Hauptverstärkerkurve sowie der Sprung in der Vorverstärkerkurve zu finden ist,
tragen diese ``Ableseungenauigkeiten'' kaum zum zufälligen Fehler bei und
werden vernachlässigt. Systematische Fehler könnten aufgetreten sein (z.B.
durch eine konstante Verzögerung zwischen Channel 1 und Channel 2), aber da wir
dazu keine Aussage treffen können, betrachten wir auch solche Fehler nicht
weiter. Die somit resultierenden Ergebnisse können in \tref{latenzmesswerte}
betrachtet werden.
\begin{table}[htbp]
\centering
% \setlength{\tabcolsep}{14pt}
\begin{tabular*}{\columnwidth}{%
S[tabformat=1.1]%
S[tabformat=1.1]%
S[tabformat=1.1]%
S[tabformat=1.1]%
S[tabformat=1.1]%
S[tabformat=1.1]%
S[tabformat=1.1]%
S[tabformat=1.1]%
}
\toprule
{$x_1$} & {$x_2$} & {$x_3$} & {$x_4$} & {$x_5$} & {$x_6$} & {$\bar x$} & {$Δx$}\\
\midrule
6.7 & 6,8 & 6,5 & 6.8 & 6.8 & 6.6 & 6.7 & 0.1 \\
\bottomrule
\end{tabular*}
\label{tab:latenzmesswerte}
\caption{Messwerte der Latenzzeit}
\end{table}
Die Latenzzeit beträgt somit
\begin{equation}
  Δt = \SI{6.7(1)}{\micro\second}		%müsste mit student-fak. ok sein
\end{equation}
. Der angegeben Fehler ist dabei das Vertrauensintervall und nicht die
StandardDaabweichung. Da die Öffnungszeit jedoch vom Computer vollautomatisch
gemessen wird, ist eine Latenzzeit in diesen Versuch unerheblich.

\subsection{Untergrundmessung}

\subsection{Eichung}


