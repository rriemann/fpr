\section{Voraussetzungen}

\subsection{Versuchsziel}

In diesem Versuch sollte die Lebensdauer von Myonen vermessen werden. Dafür
wurden verschiedene Logikschaltungen aufgebaut, mit denen die mit
Szintillationszählern registrierten Signale untersucht werden konnten. Aus dem
resultierenden Zeit-Intensitäts-Spektrum kann man dann die Halbwertszeit
ableiten.

\subsection{Physikalische Grundlagen}

Die hier untersuchten Teilchen stammen aus der kosmischen Höhenstrahlung. Ein
aus dem Weltall zur Erde gelangende Protonen wechselwirken mit Atomkernen der
oberen Atmosphäre, woraufhin ein Teilchenschauer entsteht, der hauptsächlich
aus Pionen besteht. Die geladenen $π^{\pm}$ wiederum zerfallen größtenteils in
$μ^{\pm}$ und die dazugehörigen (Anti-) Neutrinos:
\begin{eqnarray}
π^+ &\rightarrow& μ^+ ν_μ ?????\\
π^- &\rightarrow& 
\end{eqnarray}
Diese Myonen besitzen eine Lebensdauer von ca. \SI{2}{\micro\second} und
üblicherweise eine Geschwindigkeit, die fast der Lichtgeschwindigkeit
entspricht. Nach klassischer Rechnung
\subsection{Versuchsaufbau}



\subsection{Verwendete Geräte}

Für die Versuchsdurchführung wurde der im \cite{script} vorgeschlagenen und
bereits vorbereitete Aufbau verwendet. Eine Auflistung aller verwendeten
Geräte und Materialien befindet sich ebenfalls im \cite[Kap. 6]{script}.