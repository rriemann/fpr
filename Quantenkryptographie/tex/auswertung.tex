\section{Beschreibung der gemessenen Daten}
\subsection{Similarity}
Die Messgröße, die in unserem Experiment gemessen wurde, ist die Similarity $S$.
Diese beschreibt, welcher Anteil der gesendeten Informationen korrekt von Bob
rekonstruiert werden konnte. Dies lässt sich über folgende Relation ausdrücken:
\begin{equation}
S = \frac{N_{\mathrm{korr}}}{N_{\mathrm{ges}}},
\end{equation}
wobei $N_{\mathrm{korr}}$ die Zahl der von Bob korrekt detektierten Bits und
$N_{\mathrm{ges}}$ die Zahl der insgesamt von Alice gesendeten Photonen ist.

Die Zahl der richtig gemessenen Photonen hängt hierbei von der Wahl der
Umstände wie zum Beispiel der Abschwächung des Lichts durch die Filter oder
auch von der Laser-Puls-Frequenz. Um diese Abhängigkeiten zu untersuchen, haben
wir verschiedene Messreihen aufgenommen. Für jeweils eine eingestellte Frequenz
wurde der Dämpfungsexponent durch Hinzufügen oder Entfernen eines Filters
bekannter Stärke verändert und anschließend wurde die Similarity bestimmt. Die
zugehörigen Messdaten können in Anhang \ref{sec:messwerttabellen} betrachtet
werden.