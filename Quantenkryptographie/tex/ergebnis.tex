\section{Ergebnis}

Schließlich wurde vom verwendeten LabView-Programm nach jedem Durchlauf
automatisch ein Bild mit Hilfe von Alices Schlüssel kodiert und mit Bobs
Schlüssel wieder dechiffriert. Dabei kam es selbst bei der maximal erreichten
Similarity nie zu einer auch nur groben Ähnlichkeit von Original- und
dekodiertem Bild. Einzig die dominanten Farbtöne sind erhalten geblieben. Für
eine anwendbare Verschlüsselung muss demnach eine deutlich größere Similarity
erreicht werden.

Um diesem Ziel näher zu kommen, könnte eine hermetisch versiegelte Apparatur
verwendet werden, um Streulicht von außerhalb ausschließen zu können. Weiterhin
müsste die Kalibration der EOMs noch verfeinert werden, um mit sich einer
deterministischen Messung bei gleicher Basiswahl von Alice und Bob weiter zu
nähern. 

Die Genauigkeit der Dämpfungsexponenten und der Frequenzeinstellung könnte
ebenfalls überprüft werden und gegebenenfalls verbessert werden.

Abschließend soll noch einmal erwähnt werden, dass eine Theorie benötigt wird,
um aussagekräftige Vergleiche anstellen zu können und sinnvolle Fits an die
Daten durchführen zu können. 