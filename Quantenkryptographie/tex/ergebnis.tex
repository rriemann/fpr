\section{Ergebnis}

Schließlich wurde vom verwendeten LabView-Programm nach jedem Durchlauf
automatisch ein Bild mit Hilfe von Alice‘ Schlüssel kodiert und mit Bobs
Schlüssel wieder dechiffriert. Es wurde eine maximale Similarity von 
\SI{78,5}{\percent} bei einer Puls-Frequenz von \SI{10}{kHz} und einem 
Dämpfungsexponenten von \SI{4,5}{} bei einer Schlüssellänge von 4653 bei
\SI{20000}{} von Alice gesendeten Photonen erreicht, wobei es selbst hier nie zu einer auch nur
groben Ähnlichkeit von Original- und dekodiertem Bild kam. Einzig die dominanten
Farbtöne sind erhalten geblieben. Für eine anwendbare Verschlüsselung muss
demnach eine deutlich größere Similarity erreicht werden.

Um dem Ziel einer korrekten Datenkodierung näher zu kommen, könnte eine hermetisch versiegelte Apparatur
verwendet werden, um Streulicht von außerhalb ausschließen zu können. Weiterhin
müsste die Kalibration der EOMs noch verfeinert werden, um mit sich einer
deterministischen Messung bei gleicher Basiswahl von Alice und Bob weiter zu
nähern. 

Die Genauigkeit der Dämpfungsexponenten und der Frequenzeinstellung könnte
ebenfalls überprüft werden und gegebenenfalls verbessert werden.

Abschließend soll noch einmal erwähnt werden, dass eine Theorie benötigt wird,
um aussagekräftige Vergleiche anstellen zu können und sinnvolle Fits an die
Daten durchführen zu können. 