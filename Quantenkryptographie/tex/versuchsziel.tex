\section{Voraussetzungen}

\subsection{Versuchsziel}

Im Fortgeschrittenen-Praktikum „Quantenkryptographie“ besteht das Versuchsziel
darin, einen digitalen Schlüssel unter Verwendung des BB84-Protokolls
quantenkryptographisch verschlüsselt über eine kurze Distanz zu übertragen.
Die hierfür benötigten quantenmechanischen Zustände werden durch verschiedene
Polarisationen des zur Übertragung verwendeten Laserlichts repräsentiert.

\subsection{Motivation}

Um vertrauliche Informationen auszutauschen, ist die Verschlüsselung das
Mittel der Wahl. Der Sender wird allgemeinhin mit Alice, der Empfänger mit Bob
bezeichnet. Werden während der Überbringung oder Übertragung die Daten von
einem Dritten, oft Eve genannt, abgefangen bzw. mitgelesen, so kann dies bei
der Wahl eines klassischen Übermittlungsweges nicht bemerkt werden, im Falle
von quantenkryptographischen Methoden kann aber mit Hilfe von Vergleichen der
Ergebnisse von Sender und Empfänger über statistische Methoden die Existenz
von Eve nachgewiesen werden. In einem solchen Fall kann der ausgetauschte
Schlüssel einfach verworfen werden, die Nachricht selbst wird demnach auch
nicht übermittelt werden.

Mithilfe des sicher übertragenen Schlüssels kann nun auf öffentlichem Wege die
chiffrierte Nachricht übermittelt werden. Für jeden, der nicht im Besitz des
Schlüssels ist, ist die verschlüsselte Botschaft nicht lesbar und daher nutzlos.


\subsection{Physikalische Grundlagen}

Die quantenkryptographische Sicherheit kann mit Hilfe des BB84-Protokolls
erreicht werden. Hier wird die Polarisationsrichtung von Photonen verwendet, um
Informationen zu übertragen. Zunächst erzeugt Alice unpolarisierte Photonen mit
Einzelphotonenquellen wie beispielsweise Quantenpunkten. Diese werden durch
einen Polarisationsfilter geleitet, der die Photonen linear polarisiert. Die
Richtung dieses Filters kann dabei in \SI{45}{\degree}-Schritten von 0 bis
\SI{135}{\degree} eingestellt werden. Dabei stellen die Positionen bei null und
\SI{90}{degree} die Achsen eines „ungedrehten“, rechtwinkligen
Koordinatensystems dar, während die Einstellung bei 45 und \SI{135}{degree} die
Achsen eines gedrehten, ebenfalls rechtwinkligen Koordinatensystems darstellen.
Die beiden Koordinatensysteme repräsentieren jeweils eine Basis, bezüglich der
die Polarisation der Photonen gemessen wird. 

\subsection{Versuchsaufbau}