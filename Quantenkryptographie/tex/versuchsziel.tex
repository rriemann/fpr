\section{Voraussetzungen}

\subsection{Versuchsziel}

Im Fortgeschrittenen-Praktikum „Quantenkryptographie“ besteht das Versuchsziel
darin, einen digitalen Schlüssel unter Verwendung des BB84-Protokolls
quantenkryptographisch verschlüsselt über eine kurze Distanz zu übertragen.
Die hierfür benötigten quantenmechanischen Zustände werden durch verschiedene
Polarisationen des zur Übertragung verwendeten Laserlichts repräsentiert.

\subsection{Motivation}

Um vertrauliche Informationen auszutauschen, ist die Verschlüsselung das
Mittel der Wahl. Der Sender wird allgemeinhin mit Alice, der Empfänger mit Bob
bezeichnet. Werden während der Überbringung oder Übertragung die Daten von
einem Dritten, oft Eve genannt, abgefangen, so kann dieser mit den


\subsection{Physikalische Grundlagen}

\subsection{Versuchsaufbau}