%%%%%%%%%%%%%%%%%%%%%%%%% file draft4_2.tex %%%%%%%%%%%%%%%%%%%%%%%%%%%%%%%
%	                                                                   %
% Based on the  template file for The European Physical Journal C          %
%                                                                          %
%                                                                          %
%                                                                          %
%                                                                          % 
%%%%%%%%%%%%%%%%%%%%%%%%% Springer-Verlag %%%%%%%%%%%%%%%%%%%%%%%%%%%%%%%%%%

\documentclass[epj,nopacs]{svjour}
\usepackage{graphics}
\usepackage{epsfig}
\usepackage{latexsym}
\usepackage{amssymb}

\newcommand{\spar}{{\stackrel{\rightarrow}{\Rightarrow}}}
\newcommand{\sant}{{\stackrel{\rightarrow}{\Leftarrow}}}
\newcommand{\sperpa}{{\rightarrow\Uparrow}}
\newcommand{\sperpb}{{\rightarrow\Downarrow}}
\begin{document}
\hugehead

\title{Title, z. B. Z-Resonance at LEP}
\titlerunning{Titel, z. B. Z-Resonance at LEP}
\authorrunning{Student1, Student2}
\author{{\bf Draft Version 1.1}\\
\medskip \\
Vorname.~Student1,$^{1}$
Vorname.~Student2$^{1}$
} 
\institute{$^1$Department of Physics, Humboldt University of Berlin, Germany\\}  
\date{Received: \today / Revised version:}
\abstract{
We had a lot of fun measuring the lifetime and the mass of the Z boson! These quantities 
are important to answer the question of life, universe and everything.
Another nice thing we could access was the number of the lepton families, which we 
were especially happy about...  
}
\maketitle

%%%%%%%%%%%%%%%%%%%%%%%%%%%%%%%%%%%%%%%%%%%%%%%%%%%%%%%%%%%%%%%%%%%%%%%%%%%%%%
\vspace*{-1.5cm}
\section{ Introduction}
\baselineskip=0.38cm
\vspace*{1.cm}

The model describes...
The Z-boson is...
The model parameters are...

It was measured...
.......

\section{ Experiment}

Z can decay via ....
and can be observed at ...
The LEP is ....
The L3 detector consists of ...
.....
\section{ Data analysis}

The decay of the Z in the channel ... was studied...
The cuts used are described ...

\subsection{ Event selection}

\subsection{Extraction of the cross section}

Example of an equation:

\begin{equation}
\label{a1vm}
A_{1}^{} = \frac{A^{}_{||}}{D^{}}  - \eta \sqrt{R^{}}\ . \\
\end{equation}


\subsection{Background subtraction}

\begin{figure}[ht]
\vspace*{-0.2cm}
%\resizebox{0.5\textwidth}{0.35\textwidth}{\includegraphics{./r_06m.eps}}
\caption{\baselineskip=0.38cm Example of a picture}
\label{fig:1}   
\end{figure} 
...


\begin{table}[h]
\begin{center}
\begin{tabular}{|l|l|l|}
\hline
\multicolumn{1}{|c|}{}& proton & deuteron\\
\hline
\multicolumn{3}{|c|}{Exclusive electroproduction}\\
\hline
$A^\rho_1$       & 0.23 $\pm$ 0.13$\pm$0.01 & -0.040 $\pm$ 0.076$\pm$0.003\\ 
$A^\phi_1$       & 0.20$\pm$0.45$\pm$0.01  & 0.17$\pm$0.27$\pm$0.01\\
$N^\rho$         &1774                   & 6505\\
$N^\phi$         & 219                   & 618 \\
\hline
\multicolumn{3}{|c|}{Electroproduction by quasi-real photons}\\
\hline
$A^\rho_1$  & 0.0057 $\pm$ 0.0093$\pm$0.0004  & -0.0039 $\pm$ 0.0029$\pm$0.0003\\ 
$A^\phi_1$  & 0.052 $\pm$ 0.084$\pm$0.003  & 0.018 $\pm$ 0.028$\pm$0.001\\
$N^\rho$    & $423\times10^{3}$ & $4013\times10^{3}$\\
$N^\phi$ & $7.6\times10^{3}$ &  $57\times10^{3}$ \\
\hline
\end{tabular}
\vspace*{0.3cm}
\caption{\baselineskip=0.38cm Example of a table}
\label{tab_asy}
\end{center}
\vspace*{-0.5cm}
\end{table}


\section{ Summary}

\begin{thebibliography}{}
\baselineskip=0.38cm
\bibitem{z1} T. H. Bauer et al., Rev. Mod. Phys. \textbf{50} No 2 (1978) 261  
\bibitem{z2}D. Schildknecht et al., Phys. Lett. \textbf{B 449} (1999) 328
\bibitem{z3}A. Donnachie, P. V. Landshoff, Phys. Lett. \textbf{B 478} (2000) 146

\end{thebibliography}
\end{document}
