\documentclass[a4paper,twocolumn,11pt]{article}
% \documentclass[a4paper,twocolumn=on,11pt,DIV11,pagesize]{scrartcl}
\usepackage[pdftex]{graphicx}
\usepackage{subfigure}
\usepackage{anysize}
\usepackage{amssymb} % mathematische Symbole
\usepackage{amsmath} % erweiterte mathematische Symbole
\usepackage{siunitx}
\sisetup{
	seperr		=	true,
	trapambigerr	=	true,
	openerr		=	(,
	closeerr	=	),
	expproduct	=	cdot,
	padnumber	=	both,
	stickyper	=	true,
	per		=	slash,
	trapambigfrac	=	true,
	repeatunits	=	false,
	openfrac	=	(,
	closefrac	=	),
	prefixsymbolic 	=	true,
	prefixproduct	=	cdot,
	decimalsymbol	=	comma,
        tabnumalign     =       left,
        tabtextalign    =       left
}
\usepackage{longtable} % für mehrseitige Tabellen
\usepackage{booktabs}
% \usepackage{float} \restylefloat{figure} % wichtig für für genaues Bild-positionieren [H]
\usepackage[utf8]{inputenc}
\usepackage[T1]{fontenc}
\usepackage{ngerman}
\usepackage[pdftex]{hyperref}
\usepackage{booktabs}
\usepackage{color}
\definecolor{grey}{rgb}{.4,.4,.4}
\usepackage{listings}
\definecolor{ltgray}{gray}{0.90}
% \definecolor{darkblue}{rgb}{0,0,.6}
% \definecolor{darkred}{rgb}{.6,0,0}
\definecolor{darkgreen}{rgb}{0,.35,0}
% \definecolor{red}{rgb}{.98,0,0}
\lstset{language=C++,
%   commentstyle=\itshape\color{darkgreen},
%   commentstyle=\color{darkgreen},
%   keywordstyle=\bfseries, %\color{darkblue},
%   stringstyle=\color{darkred},
%   basicstyle=\ttfamily\scriptsize,
  morekeywords={TH1F,TLorentzVector,TVector3,cevent,vector,TFile},
  basicstyle=\scriptsize,
  numbers=left,
  numberstyle=\tiny,%\color{gray},
  stepnumber=1,
  tabsize=4,
  showspaces=false,
  showstringspaces=false,
  breaklines=true,
  frame=lrtb,
  captionpos=b,
  extendedchars=true,
  inputencoding=utf8,
%   backgroundcolor=\color{ltgray}
}
\hypersetup{
% 	colorlinks	=	true,
% 	urlcolor	=	darkblue,
	pdftitle	=	{Report zum 2. Elektronikversuch},
	pdfsubject	=	{Praktikumsversuch}
	pdfauthor	=   {robert.riemann@physik.hu-berlin.de,thomas.murach@physik.hu-berlin.de},
	pdfkeywords	=	{Fortgeschrittenen-Praktikum,Praktikumsversuch, Teilchenphysik, 2010, Z Resonanz},
	pdfcreator	=	{pdftex},
	pdfproducer	=	{pdftex}
}

\newcommand{\dd}[1]{\mathrm{d}#1\,} % declare dx operator, usage: \dd{x} for dx
\newcommand{\lref}[1]{Listing (\ref{lst:#1})} % refer to a listing, usage: \lref{label} for Listing (...)
\newcommand{\fref}[1]{Abb. (\ref{fig:#1})} % refer to a figure, usage: \lref{label} for Abb. (...)
\newcommand{\tref}[1]{Tab. (\ref{tab:#1})} % refer to a table, usage: \lref{label} for Tab. (...)
\newcommand{\eref}[1]{Gl. (\ref{eqn:#1})} % refer to a equation, usage: \lref{label} for Tab. (...)

\begin{document}
% % % % % % % % % % % % % % % % % % % % % % % % 
\title{{\centering \rule{15cm}{0.001cm}\\
\Large{\textsc{Institut für Physik der
Humboldt-Universität zu Berlin}}}\\ \centering \rule{15cm}{0.001cm}\\
\vspace{15mm} \centering
\includegraphics[scale=0.9]{siegel}\\
\vspace{18mm}
{\bf{\huge{Fortgeschrittenen-Praktikum}}}\\
Versuchsprotokoll\\
\vspace{14mm}
$Z^0$-Resonanz\\
\vspace{14mm} {\small{\textbf{Betreuer: M. zur Nedden}}}\\}
\author{Robert Riemann; Matr.Nr.: 521085\\
Thomas Murach; Matr.Nr.: 517771\vspace{18mm}}
\vspace{18mm}
% \date{15. Juni 2008}
% % % % % % % % % % % % % % % % % % % % % % % %
\onecolumn
\maketitle
\twocolumn

\tableofcontents
\listoffigures
\listoftables

\section{Voraussetzungen}

\subsection{Versuchsziel}

Ziel des Versuches $Z^0$-Resonanz ist das Nachzuvollziehen der Daten-Analyse des
L3-Experiments am LEP-Beschleuniger von 1989, aus der im Ergebnis die
Eigenschaften des damals neu entdeckten $Z^0$-Bosons, die Anzahl der
Quark-Farben und der elektroschwache Mischungswinkel hervorgehen.

\subsection{Physikalische Grundlagen}

Kollidieren ein Elektron und ein Positron, kann die inelastische Streuung durch
die elektromagnetische Wechselwirkung unter Austausch eines Photons vermittelt
werden, oder aber durch die schwache Wechselwirkung unter Austausch eines
sogenannten $Z^0$-Bosons. Bei hohen Schwerpunktsenergien $\sqrt{s}$ nahe der
$Z^0$-Masse $M_Z$ ist die elektromagnetische Wechselwirkung gegenüber der
schwachen jedoch stark unterdrückt.
Die $Z^0$-Resonanz bezeichnet den Kurven-Verlauf, der sich aus der
Berechnung des Wirkungsquerschnitts für den Zerfall
\begin{equation}
  e^{+} e^{-} \rightarrow Z^0 \rightarrow f \bar{f}
\end{equation}
ergibt und sein Maximum gerade bei einer Schwerpunktsenergie
$\sqrt{s}$, die der $Z^0$-Masse $M_Z$ entspricht, einnimmt. Die typische Form
der Breit-Wigner-Kurve resultiert hierbei aus dem entsprechenen Propagator,
wie im \cite[Gl. 2]{script} weiter ausgeführt wurde. Der Wirkungsquerschnitt für
ein Fermion $f$ mit der Zerfallsbreite des $Z^0$-Bosons $\Gamma_Z$ beträgt dann
\begin{equation}
  \label{eqn:sigmaf}
  \sigma_f = \sigma_0 \frac{s \Gamma_Z^2}{(s - M_Z^2)^2 + M_Z^2\Gamma_Z^2}
\end{equation}
mit dem Maximum $\sigma_0$ im Fall von $\sqrt{s} = M_Z$ bei
\begin{equation}
  \sigma_0 = \frac{12\pi}{M_Z^2}\frac{\Gamma_e\Gamma_f}{\Gamma_Z^2}
\end{equation}
. Hierbei ist $\Gamma_e$ die partielle Zerfallsbreite des Elektrons. Weiterhin gilt
für das Fermion mit Zerfallsbreite $\Gamma_f$ und elektrischer Ladung $Q_f$ nach
\cite[Gl. 6]{script} und Fermikonstante $G_F$ aus dem \cite{pdb}:
\begin{equation}
  \Gamma_f = \frac{G_F M_Z^3}{24 \sqrt{2}\pi}\left(1+[1-4|Q_f|\sin^2θ_W]^2\right)
\end{equation}

Bestimmt man nun experimentell Wirkungsquerschnitte der $Z_0$-Produktion in
Abhängigkeit von der Schwerpunktsenergie $\sqrt{s}$, so lassen sich mit einem
Fit an \eref{sigmaf} die Parameter $\sigma_0$, $M_Z$, $\Gamma_Z$ und mit
\begin{equation}
  \tau_Z = \frac{1}{\Gamma_Z}
\end{equation}
auch die Lebensdauer des $Z^0$-Bosons bestimmen.


\subsection{Versuchsaufbau}

\begin{figure}[htb]
 \centering
 \includegraphics[page=5,viewport=286 620 508 765,clip,%
  width=\columnwidth,keepaspectratio]{%
  ../docs/Z0ResFprakt}
 \caption{Aufbau das L3 Detektors}
 \label{fig:aufbau_detektor}
\end{figure}

Ausgangspunkt für diesen Versuch waren experimentell gemessene sowie auch
simulierte Daten. Es fand eine reine Datenauswertung statt.
Die Messdaten stammen aus dem L3-Experiment, dessen Detektor in
\fref{aufbau_detektor} schematisch dargestellt ist.

Der Detektor ist aus mehreren Schichten aufgebaut,
wie es für moderne Teilchenbeschleuniger üblich ist. Alle Bereiche besitzen näherungsweise eine Zylinderform, wobei die Zylinderachse entlang der z-Achse (das ist die Strahlachse) verläuft. Nahe am Strahlrohr befindet sich ein Spurdetektor, der in der Lage ist, die Bahnen geladener Teilchen zu verfolgen. In der Nähe des sogenannten Primär-Vertex' (also des Wechselwirkungspunktes von Elektronen und Positronen) befindet sich auch ein Vertex-Detektor, um diesen Vertex-Punkt gut auflösen zu können.

Die darauf folgende, weiter außen liegende Schicht ist das elektromagnetische Kalorimeter, das vor allem die Energie von nicht stark wechselwirkenden Teilchen, also Photonen und Elektronen, messen soll. Die Kalorimeter bestehen aus einer Vielzahl einzelner Zellen, sodass Aussagen über den Ort, an dem ein Energiebetrag deponiert wurde, möglich sind.

Dahinter befindet sich das hadronische Kalorimeter, das, wie der Name vermuten lässt, die Energiebeträge von hadronischen Teilchen messen soll. Es ist ebenfalls aus Zellen zusammengesetzt.

Ganz außen schließlich folgen die Myonenkammern. Myonen durchdringen den inneren Teil des Detektors beinahe ungehindert, sodass sie die einzigen Teilchen sind, die dort noch registriert werden können.

Man sollte noch erwähnen, dass sich auf den Zylinderflächen die Endkappen befinden. Das sind auch Kalorimeter, wobei die Reihenfolge die gleiche ist wie bei den Barrel-Kalorimetern (zuerst elektromagnetisches und dann hadronisches Kalorimeter).

In der Regel wird die genaue Impuls- und Energiemessung aus den Daten des Spurdetektors und der Kalorimeter gemeinsam rekonstruiert. Die Daten, die uns vorlagen, enthielten nur die drei Impulskomponenten eines Teilchens sowie dessen Masse. Da die einzelnen Events in getrennten Blocks vorlagen, war auch klar, wie viele Teilchen zu einem Event gehören. Die Energie lässt sich daraufhin leicht wieder gewinnen.

\section{Auswertung}

\subsection{Strahlungsbelastung}

Da für das Experiment radioaktive Präparate benutzt werden, scheint eine
Abschätzung der maximalen, persönlichen Belastung sinnvoll.

Folgende Annahmen wurden getroffen:
\begin{itemize}
  \item Die betreffende Person wiegt $m = \SI{80}{\kilo\gram}$.
  \item Die Einwirkdauer beträgt jeweils $s = \SI{12}{\hour}$.
  \item Die Aktivität $A$ aller 3 Präparate übersteigt jeweils nicht
        \SI{4e5}{\becquerel}.
  \item Die Halbzeit $t$ aller 3 Präparate wird mit dem maximalen Wert, 30 Jahren,
        abgeschätzt.
  \item Seit der Vermessung der Präparate sind aufgerundet $u = $ 3 Jahre vergangen.
  \item \SI{100}{\percent} der Strahlung wird aufgenommen.
  \item Der Strahlungsgewichtungsfaktor für γ-Strahlung beträgt 1.
  \item Die γ-Quanten haben eine maximale Energie von
        $E = \SI{0.7}{\mega\eV} = \SI{1.12e-13}{\joule}$
  \item Es gibt $n = 3$ Proben.
\end{itemize}
Damit berechnet sich die radioaktive Belastung zu
\begin{equation}
  s \left(\frac{1}{2}\right)^{u/t} n E A / m = \SI{0.07}{\milli\sievert}
\end{equation}
und macht somit bei einer durchschnittlichen Strahlungsbelastung in Deutschland von
\SI{4.3}{\milli\sievert} pro Jahr rund \SI{2}{\percent} aus.

\subsection{Signalbetrachtung}
TODO

\subsection{Latenzzeit der Messtechnik}

\begin{figure}[htb]
      \centering
      \includegraphics[width=1\columnwidth,keepaspectratio]{messverzoegerung}
      \caption{Standbild des Digital-Oszilloskops zur Messung der Latenzzeit}
      \label{fig:latenz}
\end{figure}

Durch die Verwendung des Hauptverstärkers kann das Signal mit höherer Präzision
gemessen werden, jedoch muss man dafür eine zeitliche Verzögerung $Δt$ in Kauf
nehmen. Um jene zu bestimmen, wird der Trigger des Oszilloskops auf den
Vorverstärker gestellt und das Bild manuell im richtigen Moment angehalten, so
dass im Ergebnis ein Standbild wie in \fref{latenz} exportiert werden kann.
\begin{table}[htbp]
\centering
% \setlength{\tabcolsep}{14pt}
\begin{tabular*}{\columnwidth}{%
S[tabformat=1.1]%
S[tabformat=1.1]%
S[tabformat=1.1]%
S[tabformat=1.1]%
S[tabformat=1.1]%
S[tabformat=1.1]%
S[tabformat=1.1]%
S[tabformat=1.1]%
}
\toprule
{$x_1$} & {$x_2$} & {$x_3$} & {$x_4$} & {$x_5$} & {$x_6$} & {$\bar x$} & {$Δx$}\\
\midrule
6.7 & 6,8 & 6,5 & 6.8 & 6.8 & 6.6 & 6.7 & 0.1 \\
\bottomrule
\end{tabular*}
\label{tab:latenzmesswerte}
\caption{Messwerte der Latenzzeit}
\end{table}
Die Messergebnisse sind in \tref{latenzmesswerte} enthalten. Wenn man den Fehler
der einzelnen Messung mit \num{0,1} als letzte signifikante Stelle abschätzt, so
kann der Fehler des Mittelwerts auch insgesamt großzügig mit \num{0.1} abgeschätzt
werden. Die Latenzzeit beträgt somit
\begin{equation}
  Δt = \SI{6.7(1)}{\second}
\end{equation}
. Da die Öffnungszeit jedoch vom Computer vollautomatisch gemessen wird, ist
eine Latenzzeit in diesen Versuch unerheblich.

\subsection{Untergrundmessung}

\subsection{Eichung}



\section{Diskussion der Ergebnisse}
Die fundamtentalen Ergebnisse dieses Versuchs werden in diesem Abschnitt noch einmal mit den theoretisch bzw. experimentell ermittelten Literaturwerten zu verglichen. Diese Zahlen wurden dem \cite{pdb} entnommen.\\
Zunächst haben wir die Masse des Z-Bosons bestimmt. Dieses Ergebnis weicht um gerade einmal \si{0.03}{\percent} vom gegebenen Wert ab, der \si{91.1876}{\giga\electronvolt} beträgt. Dabei muss man erwähnen, dass für die Fit-Parameter ohne die von uns nicht nachvollziehbare QCD-Korrektur sicherlich nicht so gut mit den realen Werten übereinstimmen würden.\\
Die totale Zerfallsbreite weicht vom realen Wert (\si{2.4952}{\giga\electronvolt}) um immerhin \si{4}{\percent} ab. Somit ist auch die Abweichung der Lebensdauer genauso groß. Ein Vergleich der Partialbreiten ist in \tref{partial} zu finden.
\begin{table*}[ht]
\begin{tabular*}{0.6\textwidth}{%
S[tabformat=2.1]%
l%
l}
\toprule
{$\Gamma_{l,\mathrm{theo}}$ [\si{GeV}]} &
{$\Gamma_{l,\mathrm{exp}}$ [\si{GeV}]} &
{relative Abw. [\si{\percent}]}\\
0.083 & 0.084 & 1\\
\midrule
{$\Gamma_{{\nu_l},\mathrm{theo}}$ [\si{GeV}]} &
{$\Gamma_{{\nu_l},\mathrm{exp}}$ [\si{GeV}]} &
{relative Abw. [\si{\percent}]}\\
0.166 & 0.499 & 200\\
\midrule
{$\Gamma_{\mathrm{Had,theo}}$ [\si{GeV}]} &
{$\Gamma_{\mathrm{Had,exp}}$ [\si{GeV}]} &
{relative Abw. [\si{\percent}]}\\
1.86 & 1.74 & 6\\
\bottomrule
\end{tabular*}
\caption{Zahl der analysierten Hadronen-Ereignisse und zugehörige Wirkungsquerschnitte bei verschiedenen Schwerpunktsenergien}
\label{tab:partial}
\end{table*}
Offenbar weichen die Partialbreiten nur um einige Prozent ab, allerdings gibt es große Abweichungen bei der Neutrino-Partialbreite. Allerdings ist in der Gleichung für diese Größe abgesehen von Konstanten nur die Masse einzusetzen, und die ist in unserem Experiment beinahe korrekt. Also ist diese Abweichung kein Produkt unserer Daten-Unsicherheiten, sondern eher eine Schwäche der Theorie, die zu der verwendeten Formel führt. Man sollte dabei noch erwähnen, dass  in \cite{pdb} nur eine Partialbreite für ``unsichtbare'' Events angegeben ist, wobei nicht einsichtig ist, ob dabei noch andere Ereignisse als Neutrinoevents erfasst wurden. Dies wäre natürlich die naheliegendste Erklärung für die Unstimmigkeiten.\\
Der schwache Mischungswinkel (in \cite{pdb} als $\sin^2\Theta_W$ bezeichnet) weicht um gerade einmal 0.4\si{\percent} vom gegebenen Wert ($\sin^2\Theta_W = 0.231$) ab.\\
Schließlich ist noch der Farbfaktor zu betrachten. Der Theoriewert liegt bei 3.0, während wir einen Wert von $3.2\pm 0.13$ erreichten (relative Abweichung: 6\si{\percent}). Auf jeden Fall ist 3 die sicherlich naheliegendste Antwort auf die Frage, welche Zahl von Quarkfamilien es gibt.\\
Alle Abweichungen von den realen Werten resultieren aus der Ungenauigkeit der Selektion sowie aus der Unsicherheit des Detektors selbst, was man schon an der Tatsache sehen konnte, dass es gemessene Energien von deutlich mehr als der Schwerpunktsenergie gab. Es böte sich zum Beispiel an, weitere Cut-Kriterien einzuführen, wie zum Beispiel Shape-Variablen. Damit kann man sehr gut zwischen Hadronen und nicht-Hadronen unterscheiden, allerdings übersteigt dies die Moglichkeiten innerhalb dieses Praltikumsversuchs. Wenn man also besser selektieren könnte, wäre die systematische Abweichung der Ergebnisse verbessert. Allerdings sind die statistischen Fehler oft etwas größer, sodass man hier ebenfalls Verbesserungen vornehmen müsste. Die naheliegendste Variante wäre das Verwenden einer besseren Statistik, also die Betrachtung von mehr Events. Außerdem waren die Monte Carlo-Daten nicht in verschiedene Samples für die vorliegenden Schwerpunktsenergien aufgeteilt, was natürlich ebenfalls die dominierenden Prozesse und deren quantitatives Verhältnis beeinflusst. Was sicherlich in einer tiefergehenden Studie zu betrachten wäre, ist die (bisher vernachlässigte) Zahl der Untergrundereignisse. Dies ist mit Sicherheit nicht korrekt, aber da uns keine Möglichkeit zur Verfügung stand, Informationen zu gewinnen, die über eine mutmaßende Spekulation hinausgehen, haben wir uns entschieden, den Untergrund zu vernachlässigen.
\renewcommand{\refname}{Literatur und Programme}
\begin{thebibliography}{xxxxxx}
\bibitem[Skript]{script}
Dobbert, Julia, Anleitung zum Versuch ``Leitfähigkeit und Hall-Effekt'' im Fortgeschrittenen-Praktikum, Humboldt–Universität, 2008
\bibitem[PDB]{pdb}
Particle Data Group, Particle Physics Booklet, Physics Letters, 2008
\bibitem[Barlow]{barlow}
Barlow, R. J., Statistics, Manchester University, Wiley Verlag, 1999
\bibitem[ROOT]{root}
ROOT Version 5.26/00, 2009, \href{http://root.cern.ch}{http://root.cern.ch}
\bibitem[maxima]{maxima}
Maxima Version 5.19.2, 2009, \href{http://maxima.sourceforge.net}{http://maxima.sourceforge.net}
\end{thebibliography}
\newpage

\onecolumn
\appendix
\section{Quellcode}
\lstinputlisting[label=lst:source,caption={Programmquelltext},lastline=208]{../code/main.cpp}

% \section{Beispiele}
% blabla
% 
% Weitere Information sind der Versuchsbeschreibung im \cite{script} zu entnehmen.
% 
% \begin{eqnarray}
% 	g &=& \SI{9.8157(11)}{\metre\per\second\squared} \\
%         \left[ g \right] &=& \si{\metre\per\Square\second} % großschreibung!
% 	\label{eqn:asdf}
% \end{eqnarray}
% 
% Dann kann man auf \eref{asdf} verweisen.


% \begin{figure}[htb]
% 	\centering
% 	\includegraphics[width=1\columnwidth,keepaspectratio]{Winkelabhaengigkeit2}
% 	\caption{Periodendauer in Abhängigkeit der Maximalauslenkung}
% 	\label{fig:Winkelabhaengigkeit}
% \end{figure}

% \begin{table}[htbp]
% \centering
% \setlength{\tabcolsep}{14pt}
% \begin{tabular*}{\columnwidth}{%
% S[tabformat=2.1]%
% S[tabformat=2.2]%
% S[tabformat=1.2]}
% \toprule
% {$R$ in \si{\ohm}} &
% {$U$ in \si{\volt}} &
% {$\frac{U}{U_{R=0}}$}\\
% \midrule
% \multicolumn{3}{c}{\textit{Eingangswiderstand $R_E$}}\\
% \midrule
% 0 & 16 & 1 \\
% 4.7e3 & 8 & 0.5 \\
% \midrule
% \multicolumn{3}{c}{\textit{Ausgangswiderstand $R_A$}}\\
% \midrule
% 0 & 1.8 & 1 \\
% 12 & 0.52 & 0.29 \\
% 27 & 1.2 & 0.67 \\
% 18 & 0.8 & 0.44 \\
% \bottomrule
% \end{tabular*}
% \caption{experimentelle Messung der Ein- und Ausgangswiderstände}
% \label{tab:221messungexp_widerstaende}
% \end{table}


% \newpage

% \onecolumn
% \appendix
% \begin{figure}
% 	\centering
% 	\includegraphics[width=0.98\textwidth,keepaspectratio]{Messprotokoll}
% 	\caption{Messprotokoll}
% 	\label{fig:protokoll}
% \end{figure}
%\colorbox{yellow}{} Farben verwenden
\end{document}