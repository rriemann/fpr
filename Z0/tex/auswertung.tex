\section{Bestimmung des hadronischen Wirkungsquerschnitts $\sigma_0$}
\subsection{Selektion der hadronischen Ereignisse}
Wir haben mehrere Datensamples zu Verfügung gehabt. Zunächst waren dort die gemessenen L3-Daten, die bei verschiedenen Schwerpunktenergien aufgenommen wurden (bei 89.48 GeV, 91.33 GeV und 93.02 GeV). Die integrierten Luminositäten unterscheiden sich auch und wurden auf \cite[S.9]{script} gegeben; außerdem konnten wir mit zwei Monte-Carlo-Daten arbeiten, die uns der aktuellen Theorie entsprechende Verteilungen von Hadronen und Myonen gab. Mit diesen Simulationsdaten konnten wir also überprüfen, welche Wirkung unsere Cuts auf den jeweiligen $Z^0$-Zerfallskanal hatten. So konnten wir feststellen, dass bei hadronischen Ereignissen immer mindestens ca. 9 Cluster (separierte Einträge im Kalorimeter) Teilchen detektiert haben. Daraus ließ sich ein wirksamer Cut gewinnen, der unter anderem die meisten Myonen-Ereignisse nicht zulässt. Ein weiterer Cut wurde angewendet, indem verlangt wurde, dass die (sichtbare) Energie eines Events mehr als $70\%$ der Schwerpunktsenergie beträgt. Zum Beispiel zerfallen $\tau$-Leptonen oft so, dass die Art und Anzahl der Teilchen, die in die Kalorimeter gelangt, sehr einem hadronischen Ereignis ähnelt. Allerdings wird dabei viel Energie an dabei entstehende Neutrines abgegeben, die nicht registriert werden können, sodass die sichtbare Energie in einem solchen Fall wesentlich kleiner ist. Mit diesem Cut können demnach viele $\tau$-Zerfälle ausgeschlossen werden. Dabei haben wir auch immer darauf geachtet, dass die Effizienz unserer Cuts nicht zu klein wird. Mit den eben erwähnten Cuts gelang es uns, eine Effizienz von
\begin{eqnarray}
\epsilon_{had} = \frac{9770}{10000} = 0.977
\end{eqnarray}
zu erreichen. Wenn man diese Cuts allerdings auf das Myonen-Monte-Carlo-Sample anwendet, so werden nur 19 hadronische Events erkannt (bei 9970 Events im Sample!). Die Fehlerquote durch Myonenereignisse ist also sehr gering. Unsere Schnitte sind also offenbar gut geeignet, um Hadronen-Events aus den Daten zu selektieren.

\subsection{Berechnung der hadronischen Wirkungsquerschnitte bei verschiedenen $\sqrt{s}$}
