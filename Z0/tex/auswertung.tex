\section{Bestimmung des hadronischen Wirkungsquerschnitts $\sigma_0^{Had}$}
\subsection{Selektion der hadronischen Ereignisse}
Wir haben mehrere Datensamples zu Verfügung gehabt. Zunächst waren dort die gemessenen L3-Daten, die bei verschiedenen Schwerpunktenergien aufgenommen wurden (bei 89.48 GeV, 91.33 GeV und 93.02 GeV). Die integrierten Luminositäten unterscheiden sich auch und wurden in \cite[S.9]{script} gegeben; außerdem konnten wir mit zwei Monte-Carlo-Daten arbeiten, die uns der aktuellen Theorie entsprechende Verteilungen von Hadronen und Myonen gab. Mit diesen Simulationsdaten konnten wir also überprüfen, welche Wirkung unsere Cuts auf den jeweiligen $Z^0$-Zerfallskanal hatten. So konnten wir feststellen, dass bei hadronischen Ereignissen immer mindestens ca. 9 Cluster (separierte Einträge im Kalorimeter) Teilchen detektiert haben. Daraus ließ sich ein wirksamer Cut gewinnen, der unter anderem die meisten Myonen-Ereignisse nicht zulässt. Ein weiterer Cut wurde angewendet, indem verlangt wurde, dass die (sichtbare) Energie eines Events mehr als $70\%$ der Schwerpunktsenergie beträgt. Zum Beispiel zerfallen $\tau$-Leptonen oft so, dass die Art und Anzahl der Teilchen, die in die Kalorimeter gelangt, sehr einem hadronischen Ereignis ähnelt. Allerdings wird dabei viel Energie an dabei entstehende Neutrines abgegeben, die nicht registriert werden können, sodass die sichtbare Energie in einem solchen Fall wesentlich kleiner ist. Mit diesem Cut können demnach viele $\tau$-Zerfälle ausgeschlossen werden. Dabei haben wir auch immer darauf geachtet, dass die Effizienz unserer Cuts nicht zu klein wird. Mit den eben erwähnten Cuts gelang es uns, eine Effizienz von
\begin{eqnarray}
\epsilon_{had} &=& \frac{9770 \pm \sqrt{9770} \pm 158}{10000}\\
&=& 0.98 \pm 0.01 \pm 0.02
\end{eqnarray}
zu erreichen. Dabei ist der erste angegebene Fehler der statistische Fehler und der zweite ist der systematische Fehler, der aus der ''willkürlichen'' Wahl der Schnittkriterien resultiert. Wenn man diese Cuts allerdings auf das Myonen-Monte-Carlo-Sample anwendet, so werden nur 19 hadronische Events erkannt (bei 9970 Events im Sample!). Die Fehlerquote durch Myonenereignisse ist also sehr gering. Unsere Schnitte sind also offenbar gut geeignet, um Hadronen-Events aus den Daten zu selektieren.

\subsection{Berechnung der hadronischen Wirkungsquerschnitte bei verschiedenen $\sqrt{s}$}
Die eben besprochenen Schnitte haben wir nun auf die Messdaten angewandt. Dabei kamen die in \tref{hadronic_xsecs} gezeigten Werte heraus. Die angegebenen Fehler sind wiederum statistischer und dann systematischer Fehler. Diese Konvention verwenden wir im gesamten folgenden Protokoll.\\
Der Fehler des Wirkungsquerschnitts ergibt sich durch Fehlerfortpflanzung aus den jeweiligen Fehlern von N und von $\epsilon_{had}$.

\begin{table}[htbp]
\centering
\setlength{\tabcolsep}{14pt}
\begin{tabular*}{\columnwidth}{%
S[tabformat=2.1]%
S[tabformat=2.2]%
S[tabformat=1.2]}
\toprule
{Energie / GeV} &
{Zahl N der analysierten Hadronen-Ereignisse} &
{Wirkungsquerschnitt $\sigma_0^{Had}$ / nb}\\
\midrule
89.48 & {$1848 \pm 43 \pm 52$} & {$10.55 \pm 0.18 \pm 0.16$} \\
91.33 & {$3980 \pm 63 \pm 79$} & {$29.98 \pm 0.49 \pm 0.43$} \\
93.02 & {$2149 \pm 46 \pm 51$} & {$14.56 \pm 0.24 \pm 0.24$} \\
\bottomrule
\end{tabular*}
\caption{experimentelle Messung der Ein- und Ausgangswiderstände}
\label{tab:hadronic_xsecs}
\end{table}

Diese Wirkungsquerschnitte kann man nun in das bereitgestellte Fit-Programm einsetzen. Dieses Programm wurde von uns leicht verändert und kann im Anhang unter ??BLABLA?? gefunden werden. Unter Berücksichtigung der statistischen Fehler von N, $\epsilon_{had}$ und der Luminosität (deren Fehler war in \cite[S.9]{script} mit $1\%$ des gegebenen Wertes angegeben) kann der statistische Fehler von $\sigma_0$ und mit den systematischen Fehlern von N und $\epsilon_{had}$ durch dieses Programm berechnet werden. Für den gesamten hadronischen Wirkungsquerschnitt resultiert also:
\begin{eqnarray}
\sigma_0^{Had} = 40.33 \pm 0.65 \pm 0.27
\end{eqnarray}