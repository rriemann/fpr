\section{Diskussion der Ergebnisse}
Die fundamtentalen Ergebnisse dieses Versuchs werden in diesem Abschnitt noch einmal mit den theoretisch bzw. experimentell ermittelten Literaturwerten zu verglichen. Diese Zahlen wurden dem \cite{pdb} entnommen.\\
Zunächst haben wir die Masse des Z-Bosons bestimmt. Dieses Ergebnis weicht um gerade einmal \si{0.03}{\percent} vom gegebenen Wert ab, der \si{91.1876}{\giga\electronvolt} beträgt. Dabei muss man erwähnen, dass für die Fit-Parameter ohne die von uns nicht nachvollziehbare QCD-Korrektur sicherlich nicht so gut mit den realen Werten übereinstimmen würden.\\
Die totale Zerfallsbreite weicht vom realen Wert (\si{2.4952}{\giga\electronvolt}) um immerhin \si{4}{\percent} ab. Somit ist auch die Abweichung der Lebensdauer genauso groß. Ein Vergleich der Partialbreiten ist in \tref{partial} zu finden.
\begin{table*}[ht]
\begin{tabular*}{0.6\textwidth}{%
S[tabformat=2.1]%
l%
l}
\toprule
{$\Gamma_{l,\mathrm{theo}}$ [\si{GeV}]} &
{$\Gamma_{l,\mathrm{exp}}$ [\si{GeV}]} &
{relative Abw. [\si{\percent}]}\\
0.083 & 0.084 & 1\\
\midrule
{$\Gamma_{{\nu_l},\mathrm{theo}}$ [\si{GeV}]} &
{$\Gamma_{{\nu_l},\mathrm{exp}}$ [\si{GeV}]} &
{relative Abw. [\si{\percent}]}\\
0.166 & 0.499 & 200\\
\midrule
{$\Gamma_{\mathrm{Had,theo}}$ [\si{GeV}]} &
{$\Gamma_{\mathrm{Had,exp}}$ [\si{GeV}]} &
{relative Abw. [\si{\percent}]}\\
1.86 & 1.74 & 6\\
\bottomrule
\end{tabular*}
\caption{Zahl der analysierten Hadronen-Ereignisse und zugehörige Wirkungsquerschnitte bei verschiedenen Schwerpunktsenergien}
\label{tab:partial}
\end{table*}
Offenbar weichen die Partialbreiten nur um einige Prozent ab, allerdings gibt es große Abweichungen bei der Neutrino-Partialbreite. Allerdings ist in der Gleichung für diese Größe abgesehen von Konstanten nur die Masse einzusetzen, und die ist in unserem Experiment beinahe korrekt. Also ist diese Abweichung kein Produkt unserer Daten-Unsicherheiten, sondern eher eine Schwäche der Theorie, die zu der verwendeten Formel führt. Man sollte dabei noch erwähnen, dass  in \cite{pdb} nur eine Partialbreite für ``unsichtbare'' Events angegeben ist, wobei nicht einsichtig ist, ob dabei noch andere Ereignisse als Neutrinoevents erfasst wurden. Dies wäre natürlich die naheliegendste Erklärung für die Unstimmigkeiten.\\
Der schwache Mischungswinkel (in \cite{pdb} als $\sin^2\Theta_W$ bezeichnet) weicht um gerade einmal 0.4\si{\percent} vom gegebenen Wert ($\sin^2\Theta_W = 0.231$) ab.\\
Schließlich ist noch der Farbfaktor zu betrachten. Der Theoriewert liegt bei 3.0, während wir einen Wert von $3.2\pm 0.13$ erreichten (relative Abweichung: 6\si{\percent}). Auf jeden Fall ist 3 die sicherlich naheliegendste Antwort auf die Frage, welche Zahl von Quarkfamilien es gibt.\\
Alle Abweichungen von den realen Werten resultieren aus der Ungenauigkeit der Selektion sowie aus der Unsicherheit des Detektors selbst, was man schon an der Tatsache sehen konnte, dass es gemessene Energien von deutlich mehr als der Schwerpunktsenergie gab. Es böte sich zum Beispiel an, weitere Cut-Kriterien einzuführen, wie zum Beispiel Shape-Variablen. Damit kann man sehr gut zwischen Hadronen und nicht-Hadronen unterscheiden, allerdings übersteigt dies die Moglichkeiten innerhalb dieses Praltikumsversuchs. Wenn man also besser selektieren könnte, wäre die systematische Abweichung der Ergebnisse verbessert. Allerdings sind die statistischen Fehler oft etwas größer, sodass man hier ebenfalls Verbesserungen vornehmen müsste. Die naheliegendste Variante wäre das Verwenden einer besseren Statistik, also die Betrachtung von mehr Events. Außerdem waren die Monte Carlo-Daten nicht in verschiedene Samples für die vorliegenden Schwerpunktsenergien aufgeteilt, was natürlich ebenfalls die dominierenden Prozesse und deren quantitatives Verhältnis beeinflusst. Was sicherlich in einer tiefergehenden Studie zu betrachten wäre, ist die (bisher vernachlässigte) Zahl der Untergrundereignisse. Dies ist mit Sicherheit nicht korrekt, aber da uns keine Möglichkeit zur Verfügung stand, Informationen zu gewinnen, die über eine mutmaßende Spekulation hinausgehen, haben wir uns entschieden, den Untergrund zu vernachlässigen.