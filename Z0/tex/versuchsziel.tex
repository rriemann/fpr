\section{Voraussetzungen}

\subsection{Versuchsziel}

Ziel des Versuches $Z^0$-Resonanz ist das Nachzuvollziehen der Daten-Analyse des
L3-Experiments am LEP-Beschleuniger von 1989, aus der im Ergebnis die
Eigenschaften des damals neu entdeckten $Z^0$-Bosons, die Anzahl der
Quark-Farben und der elektroschwache Mischungswinkel hervorgehen.

\subsection{Physikalische Grundlagen}

Kollidieren ein Elektron und ein Positron, kann die inelastische Streuung durch
die elektromagnetische Wechselwirkung unter Austausch eines Elektrons vermittelt
werden, oder aber durch die schwache Wechselwirkung unter Austausch eines
sogennanten $Z^0$-Bosons. Bei hohen Schwerpunktsenergien $\sqrt{s}$ nahe der
$Z^0$-Masse $M_Z$ ist die elektromagnetische Wechselwirkung gegenüber der
schwachen jedoch stark unterdrückt.

Die $Z^0$-Resonanz bezeichnet den Kurve-Verlauf, der sich aus der
Berechnung des Wirkungsquerschnitts für den Zerfall
\begin{equation}
  e^{+} e^{-} \rightarrow Z^0 \rightarrow f \bar{f}
\end{equation}
ergibt und sein Maximum gerade bei einer Schwerpunktsenergie
$\sqrt{s}$, die der $Z^0$-Masse $M_Z$ entspricht, einnimmt. Die typische Form
der Breit-Wigner-Kurve resultiert hierbei aus dem entsprechenen Propagator,
wie im \cite[Gl. 2]{script} weiter ausgeführt wurde. Der Wirkungsquerschnitt für
ein Fermion $f$ mit der Zerfallsbreite des $Z^0$-Bosons $Γ_Z$ beträgt dann
\begin{equation}
  \label{eqn:sigmaf}
  σ_f = σ_0 \frac{s Γ_Z^2}{(s − M_Z^2)^2 + M_Z^2Γ_Z^2}
\end{equation}
mit dem Maximum $σ_0$ im Fall von $\sqrt{s} = M_Z$ bei
\begin{equation}
  σ_0 = \frac{12π}{M_Z^2}\frac{Γ_eΓ_f}{Γ_Z^2}
\end{equation}
. Hierbei ist $Γ_e$ die partielle Zerfallsbreite des Elektrons. Weiterhin gilt
für das Fermion mit Zerfallsbreite $Γ_f$ und elektrischer Ladung $Q_f$ nach
\cite[Gl. 6]{script} und Fermikonstante $G_F$ aus dem \cite{pdb}:
\begin{equation}
  Γ_f = \frac{G_F M_Z^3}{24 \sqrt{2}π}\left(1+[1−4|Q_f|\sin^2θ_W]^2\right)
\end{equation}

Bestimmt man nun experimentell Wirkungsquerschnitte der $Z_0$-Produktion in
Abhängigkeit von der Schwerpunktsenergie $\sqrt{s}$, so lassen sich mit einem
Fit an \eref{sigmaf} die Parameter $M_Z$, $Γ_Z$ und mit
\begin{equation}
  τ_Z = \frac{1}{Γ_Z}
\end{equation}
auch die Lebensdauer des $Z^0$-Bosons bestimmen.





\subsection{Versuchsaufbau}