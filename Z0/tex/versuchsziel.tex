\section{Voraussetzungen}

\subsection{Versuchsziel}

Ziel des Versuches $Z^0$-Resonanz ist das Nachzuvollziehen der Daten-Analyse des
LEP-Experiments von 1989, aus der im Ergebnis die Eigenschaften des damals neu
entdeckten $Z^0$-Bosons, die Anzahl der Quark-Farben und der elektroschwache
Mischungswinkel hervorgehen.

\subsection{Physikalische Grundlagen}

Kollidieren ein Elektron und ein Positron, kann die inelastische Streuung durch
die elektromagnetische Wechselwirkung unter Austausch eines Elektrons vermittelt
werden, oder aber durch die schwache Wechselwirkung unter Austausch eines
sogennanten $Z^0$-Bosons. Bei hohen Schwerpunktsenergien $\sqrt{s}$ nahe der
$Z^0$-Masse $M_Z$



\subsection{Versuchsaufbau}