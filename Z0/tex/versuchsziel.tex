\section{Voraussetzungen}

\subsection{Versuchsziel}

Ziel des Versuches $Z^0$-Resonanz ist das Nachzuvollziehen der Daten-Analyse des
L3-Experiments am LEP-Beschleuniger von 1989, aus der im Ergebnis die
Eigenschaften des damals neu entdeckten $Z^0$-Bosons, die Anzahl der
Quark-Farben und der elektroschwache Mischungswinkel hervorgehen.

\subsection{Physikalische Grundlagen}

Kollidieren ein Elektron und ein Positron, kann die inelastische Streuung durch
die elektromagnetische Wechselwirkung unter Austausch eines Elektrons vermittelt
werden, oder aber durch die schwache Wechselwirkung unter Austausch eines
sogennanten $Z^0$-Bosons. Bei hohen Schwerpunktsenergien $\sqrt{s}$ nahe der
$Z^0$-Masse $M_Z$ ist die elektromagnetische Wechselwirkung gegenüber der
schwachen jedoch stark unterdrückt.

Die $Z^0$-Resonanz bezeichnet die Breit-Wigner-Kurve, die sich aus der
Berechnung des Wirkungsquerschnitts für den Zerfall
\begin{equation}
  e^{+} e^{-} \rightarrow Z^0 \rightarrow f \bar{f}
\end{equation}
ergibt und ihr Maximum gerade bei einer Schwerpunktsenergie
$\sqrt{s}$, die der $Z^0$-Masse $M_Z$ entspricht, einnimmt.


\subsection{Versuchsaufbau}